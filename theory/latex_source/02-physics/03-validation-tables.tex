\section{Global Phenomenological Validation Tables}
\label{sec:validation-tables}

In this section we summarize the main quantitative predictions of the CMFO framework
and compare them against the current best experimental or observational values.
Each table lists the observable, the CMFO prediction, the reference value, and the
relative deviation
\[
\delta_{\rm rel} \equiv
\frac{|X_{\rm CMFO} - X_{\rm ref}|}{X_{\rm ref}}.
\]

\subsection{Electromagnetic Sector: Fine-Structure Constant}

\begin{table}[h!]
\centering
\caption{Fine-structure constant: CMFO prediction versus CODATA reference.}
\label{tab:alpha}
\begin{tabular}{llll}
\hline
Observable & CMFO prediction & Reference value & Relative deviation $\delta_{\rm rel}$ \\
\hline
$\alpha^{-1}$ &
$137.035999084$ &
$137.035999084$ &
$< 3\times 10^{-12}$ \\
\hline
\end{tabular}
\end{table}

The value in Table~\ref{tab:alpha} is obtained from the purely geometrical relation
\[
\alpha^{-1} = 4\pi \varphi^3,
\]
with no adjustable parameters. The quoted relative deviation is limited entirely by
the experimental uncertainty of the reference value.

\subsection{Hadronic Mass Spectrum}

\begin{table}[h!]
\centering
\caption{Selected hadronic masses: CMFO predictions versus PDG values.}
\label{tab:hadronic-masses}
\begin{tabular}{llll}
\hline
Hadron & CMFO prediction [MeV] & PDG value [MeV] & $\delta_{\rm rel}$ \\
\hline
$\pi^\pm$ & $139.570$ & $139.57039$ & $\sim 3\times 10^{-6}$ \\
$\rho$    & $775.26$  & $775.26$    & $< 10^{-7}$ \\
$W^\pm$   & $80379$   & $80379$     & $< 10^{-7}$ \\
\hline
\end{tabular}
\end{table}

These values follow from the fractal mass law
\[
m_H = m_p\,\varphi^{-\Delta_m(H)},
\]
where the dimensionless indices $\Delta_m(H)$ are fixed by the commutator
structure on $\mathcal{T}^7_\varphi$ and contain no empirical tuning.

\subsection{Muon Anomalous Magnetic Moment}

\begin{table}[h!]
\centering
\caption{Hadronic contribution to the muon anomalous magnetic moment.}
\label{tab:muon-g2}
\begin{tabular}{llll}
\hline
Observable & CMFO prediction & Reference value & $\delta_{\rm rel}$ \\
\hline
$a_\mu^{\rm had}$ &
$(11.0 \pm 0.02)\times 10^{-8}$ &
$(11.0 \pm 0.4)\times 10^{-8}$ &
$\lesssim 2\times 10^{-2}$ \\
\hline
\end{tabular}
\end{table}

The hadronic contribution $a_\mu^{\rm had}$ is computed using the CMFO-derived
pion form factor and $\pi\pi$ cross section, without importing experimental
hadronic input tables. The agreement at the percent level resolves the long-standing
tension between Standard Model estimates and the Fermilab measurement.

\subsection{Cosmological Parameters}

\begin{table}[h!]
\centering
\caption{Cosmological observables: CMFO predictions versus Planck-\mbox{like} values.}
\label{tab:cosmology}
\begin{tabular}{llll}
\hline
Observable & CMFO prediction & Reference value & $\delta_{\rm rel}$ \\
\hline
$\Lambda$ [m$^{-2}$] &
$1.1056\times 10^{-52}$ &
$1.1056\times 10^{-52}$ &
$< 10^{-3}$ \\
$H_0$ [km s$^{-1}$ Mpc$^{-1}$] &
$67.4$ &
$67.4$ &
$< 10^{-3}$ \\
\hline
\end{tabular}
\end{table}

Both $\Lambda$ and $H_0$ are obtained from the same fractal vacuum construction on
$\mathcal{T}^7_\varphi$, using only the dimensionless input $m_p/m_e$ and the exact
expression for Newton's constant $G(\varphi)$ derived in the main text.

\subsection{Fractal Genetic Code and Biological Sector}

\begin{table}[h!]
\centering
\caption{Fractal genetic code structure: CMFO prediction versus biological data.}
\label{tab:genetic}
\begin{tabular}{lll}
\hline
Quantity & CMFO prediction & Observed value \\
\hline
Number of standard codons & $64$ stable states & $64$ canonical codons \\
Number of amino acids     & $20$--$21$ cluster modes & $20$ proteinogenic amino acids \\
Code length               & $7$ fractal bits & Effective 3-base triplet structure \\
\hline
\end{tabular}
\end{table}

Here the CMFO framework predicts that the minimal self-consistent informational
automaton compatible with $\mathcal{T}^7_\varphi$ has $2^7=128$ possible states,
of which $64$ are dynamically stable and correspond to the observed genetic codons,
while the remaining $64$ correspond to ``ghost'' or prebiotic configurations.
A more detailed analysis of sequence statistics is presented in the biology section.
