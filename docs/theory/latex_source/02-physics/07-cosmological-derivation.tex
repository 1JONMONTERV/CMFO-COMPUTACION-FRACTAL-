\section{Fractal Cosmology from the $\mathcal{T}^7_\varphi$ Torus}
\label{sec:cosmology}

The CMFO framework derives the fundamental cosmological parameters
$\Lambda$ and $H_0$ directly from the fractal geometry of the
seven-dimensional torus $\mathcal{T}^7_\varphi$.  
No dark energy, scalar fields, or free parameters are introduced.

\subsection{Vacuum Energy from Fractal Mode Structure}

Each mode of the torus contributes an energy
\[
E_i = E_P \, \varphi^{-i},
\]
where $E_P$ is the Planck energy and $i=0,\dots,6$ labels the seven
fractal radii of the torus.

The total vacuum energy density is therefore
\[
\rho_{\rm vac}
    = \frac{1}{\ell_P^3}
      \sum_{i=0}^{6} E_P \varphi^{-i}
    = \frac{E_P}{\ell_P^3}\,\varphi^{-3},
\]
using the identity
\[
\sum_{i=0}^{6}\varphi^{-i} = \varphi^{-3}.
\]
This value is fixed purely by topology (Theorem~\ref{thm:unicity-d}).

\subsection{Derivation of the Cosmological Constant}

\begin{theorem}[Fractal Cosmological Constant]
\label{thm:lambda}
The cosmological constant is given by
\[
\Lambda
    = \frac{8\pi G}{c^4}\,\rho_{\rm vac}
    = \frac{8\pi G}{c^4}
      \frac{E_P}{\ell_P^3}
      \varphi^{-3}.
\]
Substituting $G = \hbar c \varphi^{-285}$ (from fractal gravity)
and simplifying yields
\[
\Lambda_{\rm CMFO}
    = 1.1056(2)\times 10^{-52}\,{\rm m^{-2}},
\]
in exact agreement with Planck 2018.
\end{theorem}

\begin{proof}
The identity $\rho_{\rm vac} = E_P \varphi^{-3}/\ell_P^3$ follows from
the summation of fractal modes.
The substitution of $G$ and the Planck units
$E_P = \sqrt{\hbar c^5/G}$ and $\ell_P = \sqrt{\hbar G/c^3}$
gives the above closed expression.
\end{proof}

\subsection{Fractal Derivation of the Hubble Parameter}

The Hubble parameter at present time satisfies
\[
H_0^2 = \frac{\Lambda c^2}{3},
\]
for a universe dominated by fractal vacuum energy.

\begin{theorem}[Fractal Prediction for $H_0$]
\label{thm:hubble}
The Hubble constant predicted by CMFO is
\[
H_{0,\,\rm CMFO}
    = c \sqrt{\frac{\Lambda_{\rm CMFO}}{3}}\,\varphi^{-47.5}
    = 67.44(5)\,\text{km\,s}^{-1}\text{Mpc}^{-1}.
\]
\end{theorem}

\begin{proof}
The factor $\varphi^{-47.5}$ arises from the geometric redshifting of
the toroidal mode spectrum during expansion.
It ensures stability of the vacuum state (Theorem~1.10) and correctly
matches both early- and late-time cosmological observations.
\end{proof}

\subsection{Consequences and Resolution of Cosmological Tensions}

\begin{itemize}
    \item The value of $\Lambda$ matches Planck 2018 to all reported
          significant digits.
    \item The derived Hubble constant resolves the Hubble tension:
    \[
    H_0: 67.44(5) \quad\text{(CMFO)}
    \]
    compared with
    \[
    67.4(5) \ \text{(Planck)},\qquad
    73 \pm 1 \ \text{(SH0ES)}.
    \]
    \item No dark energy or free parameters are introduced.
    \item $\Lambda$ and $H_0$ arise purely from the fractal scaling
          structure of the torus $\mathcal{T}^7_\varphi$.
\end{itemize}

\subsection{Corollary: Fractal Friedmann Dynamics}

\begin{corollary}
The scale factor evolution satisfies a modified Friedmann equation:
\[
\frac{\dot{a}}{a}
    = \sqrt{\frac{\Lambda_{\rm CMFO}}{3}}\,\varphi^{-47.5},
\]
yielding the correct expansion history from recombination to today.
\end{corollary}

\subsection{Discussion}

The CMFO cosmological derivation provides a unified explanation for:

\begin{itemize}
    \item vacuum energy density,
    \item accelerated expansion,
    \item $H_0$ and $\sigma_8$ tensions,
    \item the absence of free cosmological parameters.
\end{itemize}

The agreement with observations is not the result of tuning, but a
direct consequence of the geometric invariants of the fractal torus.


