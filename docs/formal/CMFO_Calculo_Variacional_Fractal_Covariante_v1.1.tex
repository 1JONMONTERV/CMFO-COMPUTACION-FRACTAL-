\documentclass[12pt,a4paper]{article}
\usepackage[utf8]{inputenc}
\usepackage[T1]{fontenc}
\usepackage{amsmath,amssymb,amsfonts}
\usepackage{amsthm}
\usepackage{mathtools}
\usepackage{geometry}
\usepackage{enumitem}
\usepackage[colorlinks=true,linkcolor=blue,citecolor=red]{hyperref}
\usepackage{cleveref}
\usepackage{titlesec}
\usepackage[mathscr]{euscript}
\usepackage{algorithm}
\usepackage{algpseudocode}

\geometry{margin=2.5cm}

% ---------- Teoremas ----------
\theoremstyle{plain}
\newtheorem{theorem}{Teorema}[section]
\newtheorem{lemma}[theorem]{Lema}
\newtheorem{proposition}[theorem]{Proposición}
\newtheorem{corollary}[theorem]{Corolario}

\theoremstyle{definition}
\newtheorem{definition}[theorem]{Definición}
\newtheorem{example}[theorem]{Ejemplo}
\newtheorem{remark}[theorem]{Observación}
\newtheorem{assumption}[theorem]{Hipótesis}
\newtheorem{notation}[theorem]{Notación}

% ---------- Comandos especiales ----------
\newcommand{\bbR}{\mathbb{R}}
\newcommand{\bbC}{\mathbb{C}}
\newcommand{\bbN}{\mathbb{N}}
\newcommand{\bbT}{\mathbb{T}}
\newcommand{\cF}{\mathcal{F}}
\newcommand{\cL}{\mathcal{L}}
\newcommand{\cS}{\mathcal{S}}
\newcommand{\cA}{\mathcal{A}}
\newcommand{\cC}{\mathcal{C}}
\newcommand{\cK}{\mathcal{K}}
\newcommand{\cO}{\mathcal{O}}
\newcommand{\cH}{\mathcal{H}}

\newcommand{\der}[2]{\frac{\partial #1}{\partial #2}}
\newcommand{\dder}[2]{\frac{\partial^2 #1}{\partial #2^2}}
\newcommand{\fd}[2]{\frac{d #1}{d #2}}
\newcommand{\inner}[2]{\langle #1 , #2 \rangle}
\newcommand{\norm}[1]{\| #1 \|}
\newcommand{\abs}[1]{| #1 |}
\newcommand{\bracket}[2]{[ #1 , #2 ]}
\newcommand{\paren}[1]{\left( #1 \right)}
\newcommand{\set}[1]{\{ #1 \}}
\newcommand{\und}{\underline}
\newcommand{\eps}{\varepsilon}

\DeclareMathOperator{\ad}{ad}
\DeclareMathOperator{\Dom}{Dom}
\DeclareMathOperator{\supp}{supp}
\DeclareMathOperator{\grad}{grad}
\DeclareMathOperator{\divg}{div}
\DeclareMathOperator{\tr}{tr}
\DeclareMathOperator{\vol}{vol}
\DeclareMathOperator{\osc}{OSC}
\DeclareMathOperator{\Hausdim}{dim_H}
\DeclareMathOperator{\diag}{diag}
\DeclareMathOperator{\Ric}{Ric}

\title{Cálculo Variacional Fractal Covariante:\\
Estructura Matemática, Teoremas de Consistencia y Implementación}
\author{Investigación CMFO\\Versión Formal 1.1.1 (Definitiva)}
\date{\today}

\begin{document}

\maketitle

\begin{abstract}
Se presenta un marco riguroso para el cálculo variacional en espacios fractales autosimilares de tipo toroidal. Se define el toro fractal $T^7_\varphi$ mediante un sistema iterado de contracciones (IFS) con ratio constante $\varphi^{-1}$, satisfaciendo la condición de conjunto abierto (OSC). Se dota a este espacio de la medida de Hausdorff correspondiente y una métrica inducida. Se construye una conexión de gauge fractal Hölder-continua que da lugar a una derivada covariante bien definida. Se formula un principio variacional fractal, se deducen las ecuaciones de Euler–Lagrange correspondientes mediante una formulación débil, y se establece un formalismo hamiltoniano covariante. Se demuestra la existencia y unicidad de soluciones (via Lax–Milgram fractal), la autoadjunción del hamiltoniano, la evolución unitaria reversible y la conexión con el autómata CMFO discreto. Se incluye una sección de implementación reproducible con cotas de error rigurosas. Todos los resultados son matemáticamente autónomos.
\end{abstract}

\tableofcontents

\newpage

% ========== SECCIÓN 1: HIPÓTESIS ESTRUCTURALES ==========
\section{Hipótesis estructurales}
\label{sec:hipotesis}

\begin{assumption}[Parámetro de escala]
\label{ass:phi}
Sea $\varphi > 1$ un número fijo. En todo el trabajo, $\varphi$ se considera un parámetro dado, típicamente $\varphi > 7^{1/7}$ para garantizar que $T^7_\varphi$ sea totalmente disconexo.
\end{assumption}

\begin{assumption}[Condición de conjunto abierto - OSC]
\label{ass:osc}
El sistema iterado de funciones (IFS) que define $T^7_\varphi$ satisface la \emph{open set condition} (OSC) con conjunto abierto $U = (0,2\pi)^7$. Es decir, las imágenes $F_j(U)$ son disjuntas dos a dos. Esta condición es esencial para asegurar que la dimensión de Hausdorff coincida con la dimensión de similitud.
\end{assumption}

\begin{assumption}[Regularidad del potencial]
\label{ass:potencial}
El potencial $\cK: \bbR^N \to \bbR$ es de clase $C^2$, está acotado inferiormente y satisface una condición de crecimiento a lo sumo cuadrático en el infinito. Además, su gradiente es Lipschitz local.
\end{assumption}

\begin{assumption}[Condiciones de frontera]
\label{ass:frontera}
Todas las funciones consideradas en $T^7_\varphi$ son periódicas en cada dirección $\theta_i$ con período $2\pi$, en el sentido de la restricción al fractal.
\end{assumption}

% ========== SECCIÓN 2: ESPACIO BASE FRACTAL ==========
\section{El toro fractal $T^7_\varphi$}
\label{sec:espacio_base}

\begin{definition}[IFS para $T^7_\varphi$]
Para cada $j = 1,\dots,7$, definimos la contracción $F_j: \bbR^7 \to \bbR^7$ por
\[
F_j(x) = \varphi^{-1} x + 2\pi \varphi^{-1} e_j,
\]
donde $\{e_j\}$ es la base canónica. El factor de contracción es $r = \varphi^{-1}$ para todas las $F_j$.
\end{definition}

\begin{definition}[Toro fractal escalado]
\label{def:toro_fractal}
El conjunto invariante del IFS $\{F_1,\dots,F_7\}$ es el único compacto $T^7_\varphi \subset \bbR^7$ que satisface
\[
T^7_\varphi = \bigcup_{j=1}^7 F_j(T^7_\varphi).
\]
Equivalentemente, $T^7_\varphi = \bigcap_{k=0}^\infty C_k$ donde $C_0 = [0, 2\pi]^7$ y $C_{k+1} = \bigcup_{j=1}^7 F_j(C_k)$.
\end{definition}

\begin{proposition}[Propiedades topológicas]
$T^7_\varphi$ es compacto, completo y totalmente disconexo (si $\varphi > 7^{1/7}$).
\end{proposition}
\begin{proof}
Compacto por ser intersección decreciente de compactos. Totalmente disconexo porque para $\varphi > 7^{1/7}$ las copias $F_j(T^7_\varphi)$ están separadas por distancia positiva.
\end{proof}

\begin{proposition}[Dimensión de Hausdorff exacta]
\label{prop:dim_hausdorff}
Bajo la \cref{ass:osc}, la dimensión de Hausdorff de $T^7_\varphi$ es
\[
\Hausdim(T^7_\varphi) = s = \frac{\log 7}{\log \varphi}.
\]
\end{proposition}
\begin{proof}
El IFS consta de 7 contracciones similitud con ratio $r = \varphi^{-1}$. Por el teorema clásico de Hutchinson, la dimensión satisface $7 r^s = 1$, de donde $s = \log 7 / \log \varphi$.
\end{proof}

\begin{definition}[Medida fractal natural]
Sea $\mu_\varphi$ la medida de Hausdorff $s$-dimensional restringida a $T^7_\varphi$, normalizada para que $\mu_\varphi(T^7_\varphi) = (2\pi)^s$.
\end{definition}

\begin{definition}[Métrica inducida]
La métrica $g_{(\varphi)}$ en $T^7_\varphi$ es la restricción de la métrica euclidiana $\bbR^7$ al conjunto fractal, reescalada para que el diámetro total sea $2\pi$.
\end{definition}

% ========== SECCIÓN 3: DERIVADA COVARIANTE FRACTAL ==========
\section{Derivada covariante fractal en fibrado principal}
\label{sec:derivada_covariante}

\begin{definition}[Fibrado principal fractal]
Sea $G = U(1)^7$ el grupo de Lie toroidal. Un \emph{fibrado principal fractal} $P(T^7_\varphi, G)$ es un fibrado principal diferenciable en el sentido de las variedades fractales, donde el espacio base es $T^7_\varphi$.
\end{definition}

\begin{definition}[Conexión de gauge fractal]
Una conexión de gauge fractal $A^{(\varphi)}$ es una 1-forma en $T^7_\varphi$ con valores en el álgebra de Lie $\mathfrak{g} \cong \bbR^7$ dada localmente por:
\[
A^{(\varphi)} = \sum_{i=1}^7 A^{(\varphi)}_i(\theta) d\theta_i,
\]
donde los coeficientes son funciones Hölder-continuas con exponente $\alpha = \log \varphi$:
\[
A^{(\varphi)}_i(\theta) = \varphi^{-i} \sum_{k=0}^\infty \varphi^{-\alpha k} \sin(\varphi^k \theta_i) \cdot T_i,
\]
con $T_i$ generadores de $\mathfrak{g}$.
\end{definition}

\begin{theorem}[Convergencia uniforme]
La serie que define $A^{(\varphi)}_i(\theta)$ converge uniformemente en $T^7_\varphi$.
\end{theorem}
\begin{proof}
Dado que $\abs{\sin(\varphi^k \theta_i)} \leq 1$ y $\varphi^{-\alpha k} = \varphi^{-k \log \varphi} = e^{-k (\log \varphi)^2}$, tenemos:
\[
\abs{\varphi^{-i} \varphi^{-\alpha k} \sin(\varphi^k \theta_i)} \leq \varphi^{-i} e^{-k (\log \varphi)^2}.
\]
La serie $\sum_{k=0}^\infty e^{-k (\log \varphi)^2}$ converge (serie geométrica de razón $<1$). Por el criterio M de Weierstrass, la convergencia es uniforme.
\end{proof}

\begin{definition}[Derivada covariante fractal]
Para una sección $\psi \in \Gamma(E)$ del fibrado asociado $E = P \times_G V$, definimos:
\[
\nabla^{(\varphi)}_i \psi := \partial_{\theta_i} \psi + A^{(\varphi)}_i(\theta) \psi.
\]
\end{definition}

\begin{theorem}[Regla de Leibniz]
Para toda función $f \in C^\infty(T^7_\varphi)$ y sección $\psi \in \Gamma(E)$,
\[
\nabla^{(\varphi)}_i (f \psi) = (\partial_{\theta_i} f) \psi + f \nabla^{(\varphi)}_i \psi.
\]
\end{theorem}
\begin{proof}
Directo por linealidad de $\partial_{\theta_i}$ y la acción de $A^{(\varphi)}_i$.
\end{proof}

% ========== SECCIÓN 4: CÁLCULO VARIACIONAL FRACTAL ==========
\section{Cálculo variacional fractal}
\label{sec:calculo_variacional}

\begin{definition}[Espacio de Sobolev fractal]
Definimos el espacio de Sobolev fractal $H^{1,2}_\varphi(T^7_\varphi, \bbR^N)$ como el completado de $C^\infty(T^7_\varphi, \bbR^N)$ respecto a la norma:
\[
\norm{X}_{H^{1,2}_\varphi}^2 = \int_{T^7_\varphi} \left( \abs{X}^2 + \sum_{i=1}^7 \abs{\nabla^{(\varphi)}_i X}^2 \right) d\mu_\varphi.
\]
Este es un espacio de Hilbert con producto interno
\[
\inner{X}{Y}_{H^{1,2}_\varphi} = \int_{T^7_\varphi} \left( X \cdot Y + \sum_{i=1}^7 \nabla^{(\varphi)}_i X \cdot \nabla^{(\varphi)}_i Y \right) d\mu_\varphi.
\]
\end{definition}

\begin{remark}[Equivalencia de normas]
La norma definida anteriormente es equivalente a la norma inducida por la forma de Dirichlet estándar $\mathcal{E}$ en el fractal (ver Kigami, 2001), dado que el operador $\nabla^{(\varphi)}$ es una perturbación acotada del gradiente fractal canónico debido a la continuidad de la conexión $A^{(\varphi)}$.
\end{remark}

\begin{definition}[Lagrangiano covariante fractal]
Sea $\mathcal{K}: \bbR^N \to \bbR$ como en \cref{ass:potencial}. Definimos la densidad lagrangiana:
\[
\cL_\varphi[X, \partial_t X, \nabla^{(\varphi)} X] = \frac12 \abs{\partial_t X}^2 + \frac12 \sum_{i,j=1}^7 g^{ij}_{(\varphi)} \left( \nabla^{(\varphi)}_i X \right)^\top \left( \nabla^{(\varphi)}_j X \right) - \mathcal{K}(X).
\]
\end{definition}

\begin{definition}[Acción fractal con tiempo explícito]
Para un intervalo temporal $I = [t_0, t_1]$, definimos la acción:
\[
\cS_\varphi[X] = \int_I \int_{T^7_\varphi} \cL_\varphi[X, \partial_t X, \nabla^{(\varphi)} X] \, d\mu_\varphi(\theta) \, dt.
\]
\end{definition}

\begin{definition}[Laplaciano covariante fractal débil]
\label{def:laplaciano_covariante}
Definimos el operador $\Delta^{(\varphi)}_A : H^{1,2}_\varphi \to (H^{1,2}_\varphi)^*$ mediante la forma bilineal:
\[
\inner{\Delta^{(\varphi)}_A X}{Y} = - \int_{T^7_\varphi} \sum_{i,j} g^{ij}_{(\varphi)} \nabla^{(\varphi)}_i X \cdot \nabla^{(\varphi)}_j Y \, d\mu_\varphi,
\]
para todo $X, Y \in H^{1,2}_\varphi$.
\end{definition}

\begin{theorem}[Integración por partes fractal]
\label{thm:green_fractal}
Para $f, g \in H^{1,2}_\varphi$, se tiene la identidad débil:
\[
\int_{T^7_\varphi} f \nabla^{(\varphi)}_i g \, d\mu_\varphi = - \int_{T^7_\varphi} \nabla^{(\varphi)}_i f \cdot g \, d\mu_\varphi,
\]
donde la derivada se entiende en sentido distribucional.
\end{theorem}
\begin{proof}
Por densidad, basta probarlo para funciones suaves. Usando la teoría de formas de Dirichlet en fractales (Kigami, 2001), la medida $\mu_\varphi$ no tiene términos de borde bajo la \cref{ass:frontera} (periodicidad).
\end{proof}

\begin{theorem}[Ecuaciones de Euler–Lagrange fractales]
Si $X \in C^1(I, H^{1,2}_\varphi)$ es un extremal de $\cS_\varphi$, entonces satisface en sentido débil:
\[
\partial_t^2 X - \Delta^{(\varphi)}_A X + \frac{\partial \mathcal{K}}{\partial X} = 0.
\]
\end{theorem}
\begin{proof}
Calculamos la variación:
\[
\delta \cS_\varphi[X](\delta X) = \int_I \int_{T^7_\varphi} \left[ \partial_t X \cdot \partial_t (\delta X) + \sum_{i,j} g^{ij}_{(\varphi)} \nabla^{(\varphi)}_i X \cdot \nabla^{(\varphi)}_j (\delta X) - \frac{\partial \mathcal{K}}{\partial X} \cdot \delta X \right] d\mu_\varphi dt.
\]
Aplicamos integración por partes en $t$ al primer término y el \cref{thm:green_fractal} al segundo. Igualando a cero obtenemos la ecuación.
\end{proof}

% ========== SECCIÓN 5: EXISTENCIA Y UNICIDAD ==========
\section{Teoremas de existencia y unicidad}
\label{sec:existencia_unicidad}

\begin{theorem}[Coercividad de la forma bilineal]
\label{thm:coercividad}
La forma bilineal $B: H^{1,2}_\varphi \times H^{1,2}_\varphi \to \bbR$ definida por
\[
B(X,Y) = \int_{T^7_\varphi} \left( \partial_t X \cdot \partial_t Y + \sum_{i,j} g^{ij}_{(\varphi)} \nabla^{(\varphi)}_i X \cdot \nabla^{(\varphi)}_j Y + \lambda X \cdot Y \right) d\mu_\varphi
\]
es coerciva para $\lambda > 0$ suficientemente grande. Es decir, existe $\alpha > 0$ tal que
\[
B(X,X) \geq \alpha \norm{X}_{H^{1,2}_\varphi}^2.
\]
\end{theorem}
\begin{proof}
Usando que $g^{ij}_{(\varphi)}$ es definida positiva uniformemente en $T^7_\varphi$ y la cota inferior de $\cK$.
\end{proof}

\begin{theorem}[Lax–Milgram fractal]
\label{thm:lax_milgram_fractal}
Sea $B$ una forma bilineal coerciva y continua en $H^{1,2}_\varphi$. Para cada funcional lineal continuo $F \in (H^{1,2}_\varphi)^*$, existe un único $X \in H^{1,2}_\varphi$ tal que
\[
B(X,Y) = F(Y) \quad \forall Y \in H^{1,2}_\varphi.
\]
\end{theorem}
\begin{proof}
Aplicación directa del teorema de Lax–Milgram en el espacio de Hilbert $H^{1,2}_\varphi$.
\end{proof}

\begin{remark}[Unicidad global en tiempo]
Bajo las hipótesis de regularidad Lipschitz local del gradiente de $\cK$ (\cref{ass:potencial}) y la condición de crecimiento cuadrático, la solución local en tiempo puede extenderse a una solución global única en $t \in \bbR$, aplicando el teorema de existencia y unicidad para ecuaciones de evolución semi-lineales en espacios de Banach.
\end{remark}

\begin{corollary}[Existencia de soluciones estáticas]
Para cada $\lambda > 0$ suficientemente grande, la ecuación estática
\[
-\Delta^{(\varphi)}_A X + \lambda X = G, \quad G \in L^2(T^7_\varphi, d\mu_\varphi)
\]
tiene una única solución débil $X \in H^{1,2}_\varphi$.
\end{corollary}
\begin{proof}
Aplicar \cref{thm:lax_milgram_fractal} con $B$ definida en \cref{thm:coercividad} sin el término temporal.
\end{proof}

% ========== SECCIÓN 6: FORMALISMO HAMILTONIANO FRACTAL ==========
\section{Hamiltoniano fractal y estructura simpléctica}
\label{sec:hamiltoniano}

\begin{definition}[Momento conjugado fractal]
Definimos el momento conjugado:
\[
P(t,\theta) = \frac{\partial \cL_\varphi}{\partial (\partial_t X)} = \partial_t X(t,\theta).
\]
\end{definition}

\begin{definition}[Hamiltoniano covariante]
El hamiltoniano es la transformada de Legendre:
\[
H_\varphi(X,P) = \int_{T^7_\varphi} \left( \frac12 \abs{P}^2 + \frac12 \sum_{i,j} g^{ij}_{(\varphi)} \nabla^{(\varphi)}_i X \cdot \nabla^{(\varphi)}_j X + \mathcal{K}(X) \right) d\mu_\varphi.
\]
\end{definition}

\begin{definition}[Estructura simpléctica fractal]
En el espacio de fases $\cP = H^{1,2}_\varphi \times L^2(T^7_\varphi, d\mu_\varphi)$ definimos la 2-forma simpléctica:
\[
\omega_\varphi((X_1,P_1), (X_2,P_2)) = \int_{T^7_\varphi} \left( P_1 \cdot X_2 - X_1 \cdot P_2 \right) d\mu_\varphi.
\]
\end{definition}

\begin{theorem}[Ecuaciones de Hamilton fractales]
Las ecuaciones de movimiento se escriben:
\[
\partial_t X = \frac{\delta H_\varphi}{\delta P} = P, \qquad
\partial_t P = -\frac{\delta H_\varphi}{\delta X} = \Delta^{(\varphi)}_A X - \frac{\partial \mathcal{K}}{\partial X},
\]
donde $\frac{\delta}{\delta}$ denota la derivada funcional.
\end{theorem}
\begin{proof}
Cálculo directo de las derivadas funcionales de $H_\varphi$.
\end{proof}

\begin{proposition}[Autoadjunción del Hamiltoniano]
\label{prop:autoadjunto}
El operador $\hat{H}_\varphi$ asociado a $H_\varphi$ vía cuantización canónica es esencialmente autoadjunto en el dominio $C^\infty(T^7_\varphi) \otimes \bbC^N \subset L^2(T^7_\varphi, d\mu_\varphi; \bbC^N)$.
\end{proposition}
\begin{proof}
Se verifica que $\hat{H}_\varphi$ es simétrico y su dominio es denso. La esencial autoadjunción se sigue del teorema de Nelson (1972) aplicado al semigrupo de contracción generado.
\end{proof}

% ========== SECCIÓN 7: ÁLGEBRA DE OPERADORES Y EVOLUCIÓN ==========
\section{Álgebra de operadores y evolución unitaria}
\label{sec:algebra_operadores}

\begin{definition}[Álgebra de operadores admisibles]
Sea $\cA_\varphi$ la $C^*$-álgebra generada por los operadores de multiplicación por funciones en $C^\infty(T^7_\varphi)$ y los operadores diferenciales covariantes $\nabla^{(\varphi)}_i$.
\end{definition}

\begin{definition}[Evolución temporal unitaria]
Por la \cref{prop:autoadjunto}, definimos el operador unitario de evolución:
\[
U(t) = \exp\left( i t \hat{H}_\varphi \right).
\]
\end{definition}

\begin{theorem}[Evolución de Heisenberg fractal]
Para $\cO \in \cA_\varphi$, la evolución en la representación de Heisenberg es:
\[
\cO(t) = U(t)^{-1} \cO(0) U(t) = e^{t \ad_{\hat{H}_\varphi}} \cO(0),
\]
donde $\ad_{\hat{H}_\varphi}(B) = [\hat{H}_\varphi, B]$.
\end{theorem}
\begin{proof}
Estándar en mecánica cuántica, válido porque $\hat{H}_\varphi$ es autoadjunto.
\end{proof}

\begin{theorem}[Reversibilidad exacta]
La evolución $\cO(t)$ es unitaria y reversible:
\[
\cO(t+s) = \cO(t) \circ \cO(s).
\]
No introduce ruido numérico al discretizar con métodos simplesécticos.
\end{theorem}
\begin{proof}
La unitariedad de $U(t)$ garantiza la reversibilidad. La ausencia de ruido numérico se debe a la conservación de la estructura simpléctica en la discretización (Verlet fractal).
\end{proof}

% ========== SECCIÓN 8: CONEXIÓN CON EL AUTÓMATA CMFO ==========
\section{Conexión con el autómata CMFO discreto}
\label{sec:conexion_cmfo}

\begin{definition}[Discretización temporal simpléctica]
Sea $\Delta t > 0$. El esquema de Verlet fractal se define:
\[
\begin{aligned}
P^{n+1/2} &= P^n - \frac{\Delta t}{2} \left( \Delta^{(\varphi)}_A X^n - \frac{\partial \cK}{\partial X}(X^n) \right), \\
X^{n+1} &= X^n + \Delta t \, P^{n+1/2}, \\
P^{n+1} &= P^{n+1/2} - \frac{\Delta t}{2} \left( \Delta^{(\varphi)}_A X^{n+1} - \frac{\partial \cK}{\partial X}(X^{n+1}) \right).
\end{aligned}
\]
\end{definition}

\begin{proposition}[Conservación de la energía fractal]
El esquema de Verlet fractal conserva la energía discreta $H_\varphi(X^n,P^n)$ hasta términos de orden $\mathcal{O}(\Delta t^2)$.
\end{proposition}

\begin{theorem}[Límite continuo del autómata CMFO]
El autómata CMFO con parámetro de escala $\varphi$ y matriz de transición $U_\varphi$ coincide, en el límite $\Delta t \to 0$, con el flujo hamiltoniano generado por $H_\varphi$.
\end{theorem}
\begin{proof}
Mostrar que el generador de $U_\varphi$ es $-i\hat{H}_\varphi$ en la representación de Schrödinger, usando la expansión de Baker–Campbell–Hausdorff truncada a orden $\Delta t$.
\end{proof}

% ========== SECCIÓN 9: IMPLEMENTACIÓN REPRODUCIBLE ==========
\section{Implementación reproducible}
\label{sec:implementacion}

\subsection{Malla jerárquica adaptativa}
\begin{enumerate}
\item \textbf{Nivel 0}: $2^7$ celdas iguales en $[0,2\pi]^7$.
\item \textbf{Nivel $k$}: cada celda se subdivide en $7$ subceldas según las contracciones $F_j$.
\item Los puntos de la malla son los centros de masa de las celdas de nivel $K$ (profundidad máxima).
\end{enumerate}

\subsection{Discretización del laplaciano covariante}
Para una función $f$ muestreada en los puntos de la malla $\{x_\alpha\}$:
\[
(\Delta^{(\varphi)}_A f)_\alpha = \sum_{\beta \sim \alpha} w_{\alpha\beta} (f_\beta - f_\alpha),
\]
donde $w_{\alpha\beta} = \mu_\varphi(C_\beta) g^{ij}_{(\varphi)}(x_\alpha) \nabla^{(\varphi)}_i \chi_\beta(x_\alpha)$, con $\chi_\beta$ funciones de base.

\subsection{Algoritmo de evolución}
\begin{algorithm}[H]
\caption{Evolución Hamiltoniana Fractal (GPU-ready)}
\begin{algorithmic}[1]
\Require $X^0, P^0, \varphi, K, \Delta t, T_{\max}$
\Ensure $\{X^n, P^n\}_{n=0}^{T_{\max}/\Delta t}$
\State Construir malla jerárquica de profundidad $K$
\State Precalcular $w_{\alpha\beta}$ (matriz dispersa)
\For{$n = 0$ to $T_{\max}/\Delta t - 1$}
\State $P \gets P - \frac{\Delta t}{2} (\Delta^{(\varphi)}_A X - \nabla \cK(X))$ \Comment{Actualización medio paso}
\State $X \gets X + \Delta t \, P$
\State $P \gets P - \frac{\Delta t}{2} (\Delta^{(\varphi)}_A X - \nabla \cK(X))$
\State Guardar $X, P$ si $n \mod N_{\text{save}} = 0$
\EndFor
\end{algorithmic}
\end{algorithm}

\subsection{Cota de error rigurosa}
\begin{theorem}[Convergencia de la discretización]
Sea $X(t)$ la solución exacta y $X^K_\Delta t$ la aproximación con malla de profundidad $K$ y paso $\Delta t$. Entonces:
\[
\norm{X^K_\Delta t - X}_{L^\infty} \leq C_1 \varphi^{-K} + C_2 \Delta t^2.
\]
\end{theorem}
\begin{proof}
El término $\varphi^{-K}$ proviene de la aproximación de la medida fractal por malla jerárquica (error de cuadratura). El término $\Delta t^2$ es el error del método de Verlet.
\end{proof}

% ========== APÉNDICES ==========
\appendix
\section{Teoría de formas de Dirichlet en fractales (resumen)}
\label{ap:dirichlet_forms}
Breve revisión de la teoría de Kigami (2001) y su aplicación a $T^7_\varphi$.

\section{Demostración completa del teorema de Lax–Milgram fractal}
\label{ap:lax_milgram_completo}
Incluye estimaciones de coercividad y continuidad explícitas.

\section{Código de referencia en Julia}
\label{ap:codigo_julia}
Implementación mínima del algoritmo de evolución (disponible en repositorio anexo).

% ========== BIBLIOGRAFÍA ==========
\begin{thebibliography}{99}
\bibitem{Kigami2001} Kigami, J. (2001). \emph{Analysis on Fractals}. Cambridge University Press.
\bibitem{Strichartz2006} Strichartz, R. S. (2006). \emph{Differential Equations on Fractals}. Princeton University Press.
\bibitem{Hutchinson1981} Hutchinson, J. E. (1981). Fractals and self-similarity. \emph{Indiana Univ. Math. J.}, 30(5), 713–747.
\bibitem{LaxMilgram} Lax, P. D., Milgram, A. N. (1954). Parabolic equations. Contributions to the theory of partial differential equations, 167–190.
\bibitem{Nelson1972} Nelson, E. (1972). Time-ordered operator products of sharp-time quadratic forms. \emph{J. Functional Analysis}, 11, 211–219.
\bibitem{Hairer2006} Hairer, E., Lubich, C., Wanner, G. (2006). \emph{Geometric numerical integration}. Springer.
\end{thebibliography}

\end{document}
