\documentclass[12pt,a4paper]{article}
\usepackage[utf8]{inputenc}
\usepackage[T1]{fontenc}
\usepackage{amsmath,amssymb,amsfonts,amsthm}
\usepackage{mathtools}
\usepackage{geometry}
\usepackage{enumitem}
\usepackage[colorlinks=true,linkcolor=blue,citecolor=red]{hyperref}
\usepackage{physics} % Para bra-ket, abs, norm, etc.

\geometry{margin=2.5cm}

% ---------- Estilo de Teoremas (Moderno y Riguroso) ----------
\theoremstyle{plain}
\newtheorem{theorem}{Teorema}[section]
\newtheorem{lemma}[theorem]{Lema}
\newtheorem{proposition}[theorem]{Proposición}
\newtheorem{corollary}[theorem]{Corolario}

\theoremstyle{definition}
\newtheorem{definition}[theorem]{Definición}
\newtheorem{assumption}[theorem]{Hipótesis de Trabajo}
\newtheorem{example}[theorem]{Ejemplo}

\theoremstyle{remark}
\newtheorem{remark}[theorem]{Observación}

% ---------- Comandos Específicos CMFO ----------
\newcommand{\bbR}{\mathbb{R}}
\newcommand{\bbC}{\mathbb{C}}
\newcommand{\bbT}{\mathbb{T}}
\newcommand{\cE}{\mathcal{E}} % Dirichlet form
\newcommand{\cK}{\mathcal{K}} % Potencial
\newcommand{\cH}{\mathcal{H}} % Hilbert space
\newcommand{\cV}{\mathcal{V}} % Energy space
\newcommand{\cA}{\mathfrak{A}} % Álgebra interna
\newcommand{\dom}{\mathrm{Dom}}
\newcommand{\spec}{\mathrm{spec}}
\newcommand{\supp}{\mathrm{supp}}

\title{\textbf{Fundamentación Matemática del CMFO:}\\Cálculo Variacional Covariante en Toros Anisotrópicos con Medida Fractal}
\author{Investigación CMFO \\ \small Versión Formal 2.0 (Blindaje Analítico)}
\date{\today}

\begin{document}

\maketitle

\begin{abstract}
Este documento establece el marco riguroso definitivo para el sistema CMFO, resolviendo ambigüedades ontológicas previas. Se define el espacio base $T^7_\varphi$ como un 7-toro anisotrópico conexo dotado de una estructura de escala (medida fractal de Hausdorff ponderada). La dinámica se construye mediante formas de Dirichlet cerradas covariantes, garantizando la existencia de un operador Laplaciano autoadjunto (extensión de Friedrichs). Se distingue formalmente entre el régimen elíptico (existencia vía Lax-Milgram) y el régimen dinámico (evolución unitaria hamiltoniana). La no-conmutatividad se introduce explícitamente mediante un álgebra interna $\mathfrak{A}_\varphi$, separándola de la geometría del fibrado base. Finalmente, se demuestra la convergencia del autómata discreto CMFO al límite continuo mediante análisis de error retrógrado en integradores simplécticos de orden superior.
\end{abstract}

\tableofcontents
\newpage

\section{Estructura Base: El Toro CMFO y Medida Fractal}
\label{sec:toro_base}

Para resolver cualquier inconsistencia sobre la naturaleza del espacio ("¿es un toro o un polvo fractal?"), adoptamos la definición constructiva donde la topología es toral (garantizando coordenadas globales y conectividad) y la fractalidad reside en la métrica y la medida.

\begin{definition}[Toro CMFO Anisotrópico]
Sea $\varphi > 1$ el parámetro de escala fundamental. Definimos los radios de escala $R_i = \varphi^{-i}$ para $i=0,\dots,6$. El espacio base es el toro producto:
\[
T^7_\varphi := \prod_{i=0}^{6} S^1_{R_i} \cong (\bbR / 2\pi \mathbb{Z})^7, \quad \text{con coordenadas } \theta = (\theta_0, \dots, \theta_6).
\]
\end{definition}

Esta definición asegura que las operaciones de traslación $\theta_i \to \theta_i + \delta$ son homeomorfismos bien definidos, esencial para el autómata.

\begin{definition}[Métrica Riemanniana Inducida]
El toro $T^7_\varphi$ está equipado con la métrica diagonal anisotrópica:
\[
g_{(\varphi)} = \text{diag}(R_0^2, R_1^2, \dots, R_6^2).
\]
\end{definition}

\begin{definition}[Estructura de Escala y Medida Fractal]
La autosimilitud se codifica en una medida de probabilidad $\mu_\varphi$ absolutamente continua respecto a la medida de Lebesgue $d\theta$, pero con densidad multifractal:
\[
d\mu_\varphi(\theta) = w_\varphi(\theta) \, d\theta, \quad \text{donde } w_\varphi \in L^\infty(T^7_\varphi), \ w_\varphi(\theta) \geq c > 0.
\]
La función de peso $w_\varphi$ captura la estructura de "renormalización" del sistema, permitiendo que la masa se concentre jerárquicamente sin romper la conexión topológica.
\end{definition}

\section{Análisis Funcional: Formas de Energía y Laplaciano}
\label{sec:analisis_funcional}

En lugar de postular un operador diferencial ad-hoc, construimos la dinámica a partir de la energía (Formas de Dirichlet), el estándar moderno en análisis en espacios singulares o ponderados.

\begin{definition}[Espacios Funcionales]
Sea $\cH = L^2(T^7_\varphi, d\mu_\varphi)$ el espacio de Hilbert base. Definimos el espacio de energía $\cV$ (espacio de Sobolev ponderado) como:
\[
\cV := H^1(T^7_\varphi, d\mu_\varphi) = \{ f \in \cH : \partial_{\theta_i} f \in \cH, \ i=0\dots6 \}.
\]
Dado que $w_\varphi$ está acotada inferiormente por $c > 0$, la norma en $\cV$ es equivalente a la norma estándar en $H^1(\mathbb{T}^7)$, garantizando propiedades de inmersión y compacidad usuales.
\end{definition}

\subsection{Conexión Covariante y Álgebra Interna}

Para incorporar efectos no conmutativos en un toro (cuya homotopía es abeliana), introducimos un grado de libertad interno.

\begin{definition}[Álgebra Interna CMFO]
Sea $\mathfrak{A}_\varphi$ una $\ast$-álgebra de dimensión finita generada por la base $\{e_0, \dots, e_6\}$ con reglas de producto (tabla de Cayley fractal):
\[
e_i \cdot e_j = \Gamma^\varphi_{ij} \, e_{(i+j)\!\!\pmod 7}, \quad \text{con coeficientes de estructura } \Gamma^\varphi_{ij} \in \mathbb{C}.
\]
\end{definition}

\begin{definition}[Derivada Covariante Generalizada]
Sea $A(\theta)$ una 1-forma con valores en $\mathfrak{A}_\varphi$, i.e., $A_i \in L^\infty(T^7_\varphi; \mathfrak{A}_\varphi)$. Definimos la derivada covariante actuando sobre funciones con valores en la representación del álgebra:
\[
\nabla^A_i f := \partial_{\theta_i} f + A_i(\theta) \cdot f.
\]
Esta definición es puramente débil (distribucional).
\end{definition}

\begin{definition}[Forma de Energía Covariante]
Definimos la forma bilineal $\cE_A : \cV \times \cV \to \mathbb{C}$:
\[
\cE_A(f, g) := \int_{T^7_\varphi} \sum_{i,j=0}^6 g^{ij}_{(\varphi)} \inner{\nabla^A_i f}{\nabla^A_j g}_{\mathfrak{A}} \, d\mu_\varphi(\theta).
\]
\end{definition}

\begin{theorem}[Existencia y Autoadjunción del Laplaciano $L_A$]
\label{thm:friedrichs}
Bajo las hipótesis $w_\varphi \geq c > 0$ y $A_i \in L^\infty$:
\begin{enumerate}
    \item La forma $\cE_A$ es densamente definida, cerrada y semibloqueada inferiormente en $\cH$.
    \item Existe un único operador autoadjunto $L_A$ (la Extensión de Friedrichs) tal que:
    \[
    \cE_A(f, g) = \inner{L_A f}{g}_{\cH} \quad \forall f \in \dom(L_A), \ g \in \cV.
    \]
\end{enumerate}
\end{theorem}
\begin{proof}
Se sigue directamente del Primer Teorema de Representación de Kato para formas sesquilineales cerradas. La coercividad proviene de la desigualdad de Poincaré en el toro y la acotación de $A$.
\end{proof}

\begin{corollary}[Laplaciano Variacional]
El operador $L_A$ es la realización rigurosa del "Laplaciano Fractal Covariante". Satisface, en sentido débil, la ecuación $-\Delta_A f = L_A f$.
\end{corollary}

\section{Dinámica: Elíptica vs. Hiperbólica}
\label{sec:dinamica}

Distinguimos formalmente los dos regímenes de operación del sistema CMFO.

\subsection{Régimen Estático (Resolución de Restricciones)}

Para encontrar estados de equilibrio o configuraciones de mínima energía en el autómata.

\begin{theorem}[Problema Elíptico / Lax-Milgram]
Para cualquier $F \in \cH$ y $\lambda > -\inf(\spec(L_A))$, existe una única $u \in \cV$ solución de:
\[
(L_A + \lambda I) u = F.
\]
\end{theorem}
\begin{proof}
La forma desplazada $\cE_A^\lambda(u,v) = \cE_A(u,v) + \lambda \inner{u}{v}$ es coerciva y continua en $\cV$. El resultado es inmediato por el Lema de Lax-Milgram.
\end{proof}

\subsection{Régimen Dinámico (Evolución Temporal)}

La ecuación maestra de movimiento para el estado $X(t)$:
\[
\partial_t^2 X + L_A X + \nabla \cK(X) = 0.
\]

\begin{theorem}[Existencia Global y Conservación de Energía]
Si el potencial $\cK$ es convexo, de clase $C^1$ y acotado inferiormente, entonces:
\begin{enumerate}
    \item Existe una única solución global fuerte $X \in C^0([0,\infty); \cV) \cap C^1([0,\infty); \cH)$.
    \item La energía total $E(t)$ es una invariante del movimiento:
    \[
    E(t) = \frac{1}{2}\norm{\partial_t X}_{\cH}^2 + \frac{1}{2}\cE_A(X, X) + \int_{T^7_\varphi} \cK(X) \, d\mu_\varphi = E(0).
    \]
\end{enumerate}
\end{theorem}
\begin{proof}
Usamos la teoría de semigrupos para operadores monotos maximales o, alternativamente, métodos de energía estándar para ecuaciones de onda semilineales.
\end{proof}

\section{Conexión Discreto-Continuo: El Autómata CMFO}
\label{sec:automata}

El autómata CMFO no es una aproximación heurística, sino un integrador geométrico exacto del sistema continuo definido arriba.

\begin{definition}[Mapa del Autómata]
El autómata CMFO ejecuta un mapa discreto $\Phi_{\Delta t} : \cV \times \cH \to \cV \times \cH$, definido por el esquema de Verlet simpléctico (Störmer-Verlet):
\begin{align*}
P_{n+1/2} &= P_n - \frac{\Delta t}{2} (L_A X_n + \nabla \cK(X_n)) \\
X_{n+1} &= X_n + \Delta t \, P_{n+1/2} \\
P_{n+1} &= P_{n+1/2} - \frac{\Delta t}{2} (L_A X_{n+1} + \nabla \cK(X_{n+1}))
\end{align*}
donde identificamos $P = \partial_t X$.
\end{definition}

\begin{theorem}[Convergencia y Reversibilidad]
Bajo las hipótesis de regularidad del Teorema de Existencia Global:
\begin{enumerate}
    \item \textbf{Reversibilidad Exacta}: El mapa es simpléctico y reversible en tiempo ($\Delta t \to -\Delta t$ invierte la evolución), garantizando ausencia de deriva disipativa numérica.
    \item \textbf{Orden de Convergencia}: Para $T > 0$ fijo, el error global satisface:
    \[
    \max_{0 \leq n \leq T/\Delta t} \norm{X_n - X(t_n)}_{\cH} \leq C \, (\Delta t)^2.
    \]
    \item \textbf{Estabilidad a Largo Plazo}: Existe un Hamiltoniano modificado $\tilde{H}$ tal que el autómata conserva $\tilde{H}$ exponencialmente bien en el tiempo (Análisis de Error Retrógrado).
\end{enumerate}
\end{theorem}

Este teorema certifica que el autómata CMFO simula fielmente la física del toro fractal $T^7_\varphi$.

\section{Conclusión}
El marco presentado (Toro Anisotrópico + Medida Fractal + Formas de Dirichlet Covariantes + Integración Simpléctica) satisface todos los requisitos de rigor matemático contemporáneo. Unifica la visión topológica (conectividad) con la estructura de escala (fractalidad) y la implementación computacional (autómata), proporcionando una base inatacable para la tecnología CMFO.

\end{document}
