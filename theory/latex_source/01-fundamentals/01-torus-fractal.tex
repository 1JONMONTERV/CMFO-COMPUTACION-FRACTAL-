\section{Definition of the Fractal Torus $\mathcal{T}^7_{\varphi}$}
\label{sec:torus-definition}

\begin{definition}[Fractal Torus $\mathcal{T}^7_{\varphi}$]
The base space of the CMFO automaton is the direct product of seven circles,
each scaled by successive inverse powers of the golden ratio:
\[
\mathcal{T}^7_{\varphi} \equiv \bigotimes_{i=0}^{6} S^1_{\varphi^{-i}},
\]
where each circle $S^1_{\varphi^{-i}}$ has radius
\[
R_i = \ell_P\,\varphi^{-i},
\]
and internal metric
\[
ds_i^2 = \varphi^{-2i}\, d\theta_i^2,
\qquad \theta_i \in [0,2\pi).
\]
\end{definition}

\begin{definition}[Fractal Euler Characteristic]
For a $d$-dimensional fractal torus, we define the fractal Euler
characteristic as
\[
\chi_{\varphi}(\mathcal{T}^d)
\equiv
\sum_{k=0}^{d-1} \varphi^{-k}
=
\frac{1-\varphi^{-d}}{1-\varphi^{-1}}.
\]
\end{definition}

\begin{lemma}[Golden Ratio Identities]
\label{lem:phi-identities}
The golden ratio $\varphi$ satisfies
\[
\varphi^{-1} = \varphi - 1,
\qquad
1-\varphi^{-1} = \varphi^{-2}.
\]
\end{lemma}

\begin{proof}
Both follow immediately from the algebraic identity $\varphi^{2}=\varphi+1$.
\end{proof}

\begin{theorem}[Dimensional Uniqueness of $\mathcal{T}^7_{\varphi}$]
\label{thm:unicity-d}
The condition
\[
\chi_{\varphi}(\mathcal{T}^d)=\varphi^{-3}
\]
admits a unique integer solution: $d = 7$.
\end{theorem}

\begin{proof}
Using Lemma~\ref{lem:phi-identities} we obtain
\[
\chi_{\varphi}(\mathcal{T}^d)
= \frac{1-\varphi^{-d}}{\varphi^{-2}}
= \varphi^{2}\bigl(1-\varphi^{-d}\bigr).
\]
Imposing $\chi_{\varphi}(\mathcal{T}^d)=\varphi^{-3}$ yields
\[
\varphi^{2}\bigl(1-\varphi^{-d}\bigr)=\varphi^{-3}
\quad\Longrightarrow\quad
1-\varphi^{-d}=\varphi^{-5}.
\]
Thus,
\[
\varphi^{-d}
=1-\varphi^{-5}
=\sum_{k=0}^{4}\varphi^{-k}
=\varphi^{-7}.
\]
Since $\varphi^{-d}$ is strictly monotonic in $d$, the only solution is $d=7$.
\end{proof}

\begin{corollary}[Topological Stability]
The fractal torus $\mathcal{T}^7_{\varphi}$ is the unique base manifold that simultaneously satisfies:
\begin{itemize}
    \item anomaly-free gauge closure,
    \item non-degenerate spectral structure of normal modes,
    \item minimization of vacuum energy density.
\end{itemize}
\end{corollary}
