% CMFO_Geometria_Espectral_No_Conmutativa.tex
\documentclass[12pt,a4paper]{article}

\usepackage[utf8]{inputenc}
\usepackage[T1]{fontenc}
\usepackage{lmodern}
\usepackage{amsmath,amssymb,amsfonts}
\usepackage{amsthm}
\usepackage{mathtools}
\usepackage{geometry}
\usepackage{enumitem}
\usepackage{microtype}
\usepackage[colorlinks=true,linkcolor=blue,citecolor=red]{hyperref}

\geometry{margin=2.5cm}

% ---------- Entornos ----------
\theoremstyle{plain}
\newtheorem{theorem}{Teorema}[section]
\newtheorem{lemma}[theorem]{Lema}
\newtheorem{proposition}[theorem]{Proposición}
\newtheorem{corollary}[theorem]{Corolario}

\theoremstyle{definition}
\newtheorem{definition}[theorem]{Definición}
\newtheorem{assumption}[theorem]{Hipótesis}
\newtheorem{remark}[theorem]{Observación}

\title{\textbf{CMFO y Geometría No Conmutativa:}\\Triplete Espectral sobre $T^7_\varphi$, Distancia de Connes y Acción Espectral\\
\large Proyecto CMFO — Nota Técnica Formal}
\author{Investigación CMFO}
\date{December 17, 2025}

\begin{document}

\maketitle

\begin{abstract}
Se propone una formalización no conmutativa de la geometría CMFO mediante un triplete espectral $(\mathcal{A}, \mathcal{H}, \mathcal{D}_\varphi)$. El objetivo práctico es doble: (i) proporcionar un lenguaje estándar para conectar CMFO con el formalismo de Connes (distancia, invariantes espectrales, acción espectral), y (ii) fijar requisitos matemáticos verificables que habiliten auditoría/certificación de implementaciones (autoadjunción, conmutadores acotados, resolvente compacto). Se presenta una construcción mínima coherente con el autómata CMFO (operador de evolución reversible $U_\varphi$) y con la derivada covariante fractal ya introducida en la capa variacional.
\end{abstract}

\tableofcontents
\newpage

\section{Datos CMFO: álgebra, Hilbert y operador tipo Dirac}

\begin{assumption}[Espacio y medida]
\label{ass:measure}
Sea $(T^7_\varphi, \mu_\varphi)$ un compacto medible como en la capa variacional CMFO, y sea $\mathcal{H} := L^2(T^7_\varphi, \mu_\varphi; \mathbb{C}^m)$ para un $m$ fijo (espinores/fracciones internas).
\end{assumption}

\begin{definition}[Álgebra de coordenadas (conmutativa)]
Tomamos $\mathcal{A} := \mathrm{Lip}(T^7_\varphi)$ (o una subálgebra densa de $C(T^7_\varphi)$). Representamos $\mathcal{A}$ en $\mathcal{H}$ por multiplicación:
\[
(\pi(a)\psi)(\theta) = a(\theta)\psi(\theta).
\]
\end{definition}

\begin{assumption}[Clifford y derivadas covariantes]
Sea $\{\gamma_i\}_{i=1}^7$ una representación matricial de $\mathrm{Cl}(7)$ en $\mathbb{C}^m$: $\gamma_i \gamma_j + \gamma_j \gamma_i = 2\delta_{ij} I$. Sea $\nabla^{(\varphi)}_i$ una colección de derivadas covariantes (en sentido débil/Dirichlet) compatibles con la conexión CMFO.
\end{assumption}

\begin{definition}[Operador tipo Dirac CMFO]
Definimos formalmente
\[
\mathcal{D}_\varphi := \sum_{i=1}^7 \varphi^{-(i-1)} \gamma_i \nabla^{(\varphi)}_i,
\]
con dominio inicial $\mathrm{Dom}(\mathcal{D}_\varphi) := \mathcal{D}_0 \subset \mathcal{H}$ denso (p.ej., funciones "cilíndricas" de la aproximación por grafos).
\end{definition}

\begin{remark}
Los factores $\varphi^{-(i-1)}$ codifican el escalamiento anisotrópico del autómata CMFO (paso elemental por dimensión). Una alternativa equivalente es absorberlos en la métrica efectiva y usar un Dirac estándar respecto a $g_{(\varphi)}$.
\end{remark}

\section{Triplete espectral: condiciones y verificación}

\begin{definition}[Triplete espectral]
Un triplete espectral es $(\mathcal{A}, \mathcal{H}, \mathcal{D})$ donde: (i) $\mathcal{A}$ es una $*$-álgebra representada en $\mathcal{H}$, (ii) $\mathcal{D}$ es autoadjunto con resolvente compacto, (iii) para todo $a \in \mathcal{A}$, el conmutador $[\mathcal{D}, \pi(a)]$ se extiende a operador acotado en $\mathcal{H}$.
\end{definition}

\begin{assumption}[Autoadjunción y resolvente compacto]
Supondremos que $\mathcal{D}_\varphi$ es esencialmente autoadjunto en $\mathcal{D}_0$ y que $(I + \mathcal{D}_\varphi^2)^{-1/2}$ es compacto.
\end{assumption}

\begin{theorem}[Acotación del conmutador para $a \in \mathrm{Lip}$]
Si $a \in \mathrm{Lip}(T^7_\varphi)$ y las derivadas $\nabla^{(\varphi)}_i a$ existen en el sentido de energía con $\|\nabla^{(\varphi)}_i a\|_{L^\infty} \le L \varphi^{i-1}$, entonces $[\mathcal{D}_\varphi, \pi(a)]$ es acotado y
\[
\| [\mathcal{D}_\varphi, \pi(a)] \| \le \sum_{i=1}^7 \varphi^{-(i-1)} \| \nabla^{(\varphi)}_i a \|_{L^\infty}.
\]
\end{theorem}
\begin{proof}
Para $\psi \in \mathcal{D}_0$,
\[
[\mathcal{D}_\varphi, \pi(a)] \psi = \sum_i \varphi^{-(i-1)} \gamma_i (\nabla^{(\varphi)}_i a) \psi,
\]
y la cota sigue por submultiplicatividad y la acotación esencial de $\nabla^{(\varphi)}_i a$.
\end{proof}

\begin{remark}[Punto crítico]
En un fractal, la clase Lip puede reemplazarse por el álgebra de funciones finita-energía. El criterio real de implementación es: computar $\nabla^{(\varphi)}_i a$ en la aproximación discreta y verificar la estabilidad de la norma.
\end{remark}

\section{Distancia de Connes y métrica efectiva CMFO}

\begin{definition}[Distancia de Connes]
Para $\theta, \eta \in T^7_\varphi$, definimos
\[
d_C(\theta, \eta) := \sup \{ |a(\theta) - a(\eta)| : a \in \mathcal{A}, \|[\mathcal{D}_\varphi, \pi(a)]\| \le 1 \}.
\]
\end{definition}

\begin{proposition}[Control Lipschitz]
Si $[\mathcal{D}_\varphi, \pi(a)]$ controla el seminorma Lipschitz inducido por la energía, entonces $d_C$ induce la topología de $T^7_\varphi$.
\end{proposition}

\begin{remark}
Este es un punto de auditoría: si $d_C$ no recupera la topología esperada (toro fractal), la definición de $\mathcal{D}_\varphi$ o la elección de $\mathcal{A}$ debe ajustarse.
\end{remark}

\section{Acción espectral y conexión con dinámica CMFO}

\begin{definition}[Acción espectral]
Sea $f: \mathbb{R}^+ \to \mathbb{R}^+$ una función test (p.ej. Schwartz) y $\Lambda > 0$ un corte. Definimos la acción espectral bosónica
\[
S_{\mathrm{spec}}(\Lambda) := \mathrm{Tr}\left( f\left( \frac{\mathcal{D}_\varphi^2}{\Lambda^2} \right) \right).
\]
\end{definition}

\begin{remark}
En el enfoque de Connes–Chamseddine, la expansión asintótica de $S_{\mathrm{spec}}$ produce invariantes geométricos (curvaturas, términos gauge, etc.). En CMFO, la interpretación operacional es: los coeficientes espectrales codifican la geometría efectiva inducida por $\varphi$ y por la conexión fractal.
\end{remark}

\subsection{Vínculo explícito con el autómata $U_\varphi$}

\begin{assumption}[Generador infinitesimal]
En el régimen de paso temporal $\Delta t \downarrow 0$, el operador de evolución del autómata se aproxima por
\[
U_\varphi(\Delta t) = \exp(-i \Delta t H_\varphi),
\]
donde $H_\varphi$ es autoadjunto y compatible con la energía (capa variacional).
\end{assumption}

\begin{proposition}[Plantilla CMFO: $H_\varphi$ como función de $\mathcal{D}_\varphi$]
Si la cinética domina y la conexión es compatible, entonces existe una función Borel $g$ tal que, al nivel de operadores, $H_\varphi \approx g(\mathcal{D}_\varphi)$ (típicamente $g(x) = x^2$ o $g(x) = \sqrt{x^2 + m^2}$).
\end{proposition}

\begin{remark}
Esta identificación es el puente para aplicaciones: cualquier pipeline CMFO implementado puede auditarse a través de (i) espectro de $\mathcal{D}_\varphi$ y (ii) consistencia entre $U_\varphi$ y $H_\varphi$ por pruebas BCH.
\end{remark}

\section{Aplicaciones CMFO con ventaja competitiva}

\begin{itemize}
    \item \textbf{IA de alto cumplimiento (calidad, auditoría, certificación)}: el triplete espectral introduce invariantes fuertes (espectro, distancia de Connes, acción espectral) que permiten especificaciones verificables y pruebas de no-regresión, ausentes en IA comercial opaca.
    \item \textbf{Cómputo reversible y verificación formal}: $U_\varphi$ unitario/reversible + control por $\mathcal{D}_\varphi$ posibilita trazabilidad matemática de la ejecución (logs como invariantes).
    \item \textbf{Criptografía / integridad}: huellas espectrales sobre grafos de nivel $K$ generan identificadores difíciles de falsificar sin reproducir la geometría.
    \item \textbf{Simulación física}: la acción espectral actúa como "funcional maestro" para derivar términos de campo (gauge/curvatura) desde estructura $\varphi$.
\end{itemize}

\section{Checklist de objeciones y respuestas técnicas}

\begin{remark}[Objeción: "No hay Dirac en un fractal"]
Se trabaja con un operador tipo Dirac definido por derivadas en sentido de energía/Dirichlet forms. El requisito verificable es: (i) autoadjunción esencial, (ii) resolvente compacto, (iii) conmutadores acotados.
\end{remark}

\begin{remark}[Objeción: "El formalismo es ad hoc"]
La construcción está alineada con condiciones estándar de tripletes espectrales y fija tests computables: normas de conmutadores, espectros discretos en aproximaciones por grafos y convergencia de invariantes.
\end{remark}

\begin{thebibliography}{9}
\bibitem{Connes1994} A. Connes. \textit{Noncommutative Geometry}. Academic Press, 1994.
\bibitem{ChamseddineConnes1997} A. H. Chamseddine, A. Connes. The spectral action principle. \textit{Commun. Math. Phys.}, 186:731–750, 1997.
\bibitem{Fukushima2011} M. Fukushima, Y. Oshima, M. Takeda. \textit{Dirichlet Forms and Symmetric Markov Processes}. de Gruyter, 2nd ed., 2011.
\bibitem{Kigami2001} J. Kigami. \textit{Analysis on Fractals}. Cambridge University Press, 2001.
\bibitem{Strichartz2006} R. S. Strichartz. \textit{Differential Equations on Fractals}. Princeton University Press, 2006.
\bibitem{Christensen2006} E. Christensen, C. Ivan. Spectral triples for AF algebras and metrics on the Cantor set. \textit{J. Operator Theory}, 56(1):17–46, 2006.
\bibitem{Cipriani2003} F. Cipriani, J.-L. Sauvageot. Derivations as square roots of Dirichlet forms. \textit{J. Funct. Anal.}, 201:78–120, 2003.
\end{thebibliography}

\end{document}
