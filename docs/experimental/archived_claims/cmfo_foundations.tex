\documentclass[12pt]{article}
\usepackage{amsmath, amssymb, amsthm}

\title{\textbf{Foundations of Continuous Modal Fractal Oscillation (CMFO)}}
\author{CMFO Research}
\date{v1.0.0 Official Canonical Document}

\begin{document}

\maketitle

\section{Introduction}
This document serves as the formal definition of the CMFO computational kernel. It unifies the axiomatic structure, boolean absorption theorems, and tensorial algebra into a single canonical reference.

\section{Axiomatic Base}
\begin{enumerate}
    \item \textbf{The Substrate:} Computation occurs in $\mathcal{F}_7 = (\mathbb{R}^7, \mathcal{G}_\varphi)$, a Hilbert space with golden-ratio metric.
    \item \textbf{Operational Closure:} The space is closed under the Fractal Product $\otimes_7$ and Resonance Sum $\oplus_\varphi$.
    \item \textbf{Determinism:} Given state $S_t$ and operator $\mathcal{T}$, $S_{t+1}$ is uniquely determined. There is no stochasticity.
\end{enumerate}

\section{The Tensor-7 Operator}
The fundamental unit of computation is the T7 Projection:
\begin{equation}
    P_{T7}(v) = \frac{v \cdot \Gamma(v) + \varphi}{1 + \varphi}
\end{equation}
This operator is contractive and guarantees convergence to an attractor basin representing semantic meaning.

\section{Boolean Absorption Theorem}
See \textit{docs/math/boolean\_absorption.tex} for the full proof that:
\begin{equation}
    \lim_{\varphi \to \infty} \text{CMFO}(\text{Logic}) \equiv \text{Boolean Algebra}
\end{equation}

\end{document}
