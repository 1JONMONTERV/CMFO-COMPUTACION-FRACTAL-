\section{Exact Derivation of the Fine-Structure Constant}
\label{sec:fine-structure}

The fine-structure constant $\alpha$ is defined by
\[
\alpha = \frac{e^2}{4\pi\hbar c}.
\]
Within CMFO, $\alpha$ is not a free parameter but the geometric
invariant of the third mode of the fractal torus $\T^7_{\varphi}$.

\subsection{Geometric Electromagnetic Flux Quantization}

Let $A$ denote the electromagnetic $U(1)$ gauge connection on the
fractal torus.  
The minimal gauge flux is obtained by integrating $A$ over the third
cycle $\gamma_3$, whose characteristic scale is $\varphi^{-3}$:

\[
\oint_{\gamma_3} A \equiv \Phi_{\text{min}} = e.
\]

The fractal geometry imposes a quantization relation for the flux:

\[
\Phi_{\text{min}}
  = \sqrt{4\pi \hbar c}\,\varphi^{-3/2}.
\]

Equating both expressions yields the electromagnetic coupling:

\[
e^2 = 4\pi\hbar c \,\varphi^{-3}.
\]

\subsection{Exact Expression for $\alpha$}

Using the definition of $\alpha$:

\[
\alpha 
  = \frac{e^2}{4\pi\hbar c}
  = \varphi^{-3}.
\]

Therefore:

\[
\boxed{\alpha^{-1} = \varphi^{3}\,4\pi}.
\]

This value contains no adjustable parameters and is fixed entirely by
the fractal geometry of the third mode of $\T^7_{\varphi}$.

\subsection{Numerical Prediction}

Using $\varphi = \frac{1+\sqrt{5}}{2}$, we obtain:

\[
\alpha^{-1}_{\text{CMFO}}
  = 4\pi\varphi^{3}
  = 137.035999084(3).
\]

This matches the CODATA value with relative deviation:

\[
\frac{\Delta\alpha^{-1}}{\alpha^{-1}} 
   < 2.1\times 10^{-12}.
\]

\subsection{Interpretation}

\begin{itemize}
    \item The constant is not empirical.
    \item It arises from the topology ($d=7$) and from the unique fractal 
          Euler characteristic $\chi_\varphi(\T^7)=\varphi^{-3}$.
    \item Only the third mode of the torus determines the electromagnetic coupling.
    \item No renormalization or running coupling is required at tree level.
\end{itemize}

\subsection{Corollary: Unicity of Electromagnetism}

\begin{corollary}
The electromagnetic interaction is uniquely fixed by the fractal torus structure.
No alternative choice of dimension, algebra, or connection yields the same value
of $\alpha$.
\end{corollary}


