\documentclass{article}
\usepackage{amsmath}

\title{Absorption of Boolean Logic in CMFO}
\author{CMFO-Universe}

\begin{document}
\maketitle

\section{Classical Boolean Logic}

Boolean logic is defined over:
\[
\mathcal{B} = \{0,1\}, \quad \{\land, \lor, \lnot\}
\]

\section{CMFO Mapping}

We define the following morphism:
\[
\Phi : \mathcal{B} \rightarrow \mathbb{R}^7
\]

\begin{align}
a \land b &\mapsto a \otimes_7 b \\
a \lor b &\mapsto a \oplus_\varphi b \\
\lnot a &\mapsto \mathcal{R}_\pi(a)
\end{align}

\section{Completeness Theorem}

\textbf{Theorem:}  
For any Boolean expression $E$, there exists a CMFO operator $T_E$ such that:
\[
\Phi(E) = T_E(\Phi(inputs))
\]

\textbf{Proof Sketch:}
\begin{itemize}
\item Tensor operators are universal approximators.
\item Boolean gates are piecewise-linear limits.
\item CMFO preserves logical separability.
\end{itemize}

\section{Conclusion}

Boolean logic is a degenerate projection of CMFO tensorial logic.

\end{document}
