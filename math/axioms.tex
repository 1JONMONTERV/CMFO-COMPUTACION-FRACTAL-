\documentclass{article}
\usepackage{amsmath, amssymb}

\title{CMFO Axiomatic Foundations}
\author{CMFO-Universe}
\date{\today}

\begin{document}
\maketitle

\section{Fundamental Axioms}

\textbf{Axiom I (Fractal Space)}  
The fundamental space of computation and physics is the 7-dimensional fractal space:
\[
\mathcal{F}_7 = (\mathbb{R}^7, \mathcal{G}_\varphi)
\]
where $\mathcal{G}_\varphi$ is a self-similar metric parameterized by the golden ratio $\varphi$.

\textbf{Axiom II (Continuity)}  
All fundamental operations are continuous. Discrete logic emerges as a limit case.

\textbf{Axiom III (Tensorial Computation)}  
All computation is represented by tensorial transformations:
\[
T : \mathbb{R}^7 \rightarrow \mathbb{R}^7
\]

\textbf{Axiom IV (Non-Booleanity)}  
Boolean logic is not primitive. Logical operations are emergent projections of tensor dynamics.

\textbf{Axiom V (Fractal Closure)}  
The composition of CMFO operators is closed:
\[
T_a \circ T_b \in \mathcal{F}_7
\]

\textbf{Axiom VI (Energy Conservation)}  
Each operation preserves or redistributes fractal energy:
\[
\|T(x)\|_{\varphi} \leq C \|x\|_{\varphi}
\]

\end{document}
